% Options for packages loaded elsewhere
\PassOptionsToPackage{unicode}{hyperref}
\PassOptionsToPackage{hyphens}{url}
%
\documentclass[
]{book}
\usepackage{amsmath,amssymb}
\usepackage{lmodern}
\usepackage{ifxetex,ifluatex}
\ifnum 0\ifxetex 1\fi\ifluatex 1\fi=0 % if pdftex
  \usepackage[T1]{fontenc}
  \usepackage[utf8]{inputenc}
  \usepackage{textcomp} % provide euro and other symbols
\else % if luatex or xetex
  \usepackage{unicode-math}
  \defaultfontfeatures{Scale=MatchLowercase}
  \defaultfontfeatures[\rmfamily]{Ligatures=TeX,Scale=1}
\fi
% Use upquote if available, for straight quotes in verbatim environments
\IfFileExists{upquote.sty}{\usepackage{upquote}}{}
\IfFileExists{microtype.sty}{% use microtype if available
  \usepackage[]{microtype}
  \UseMicrotypeSet[protrusion]{basicmath} % disable protrusion for tt fonts
}{}
\makeatletter
\@ifundefined{KOMAClassName}{% if non-KOMA class
  \IfFileExists{parskip.sty}{%
    \usepackage{parskip}
  }{% else
    \setlength{\parindent}{0pt}
    \setlength{\parskip}{6pt plus 2pt minus 1pt}}
}{% if KOMA class
  \KOMAoptions{parskip=half}}
\makeatother
\usepackage{xcolor}
\IfFileExists{xurl.sty}{\usepackage{xurl}}{} % add URL line breaks if available
\IfFileExists{bookmark.sty}{\usepackage{bookmark}}{\usepackage{hyperref}}
\hypersetup{
  pdftitle={R Eğitim},
  pdfauthor={Melike Dönertaş},
  hidelinks,
  pdfcreator={LaTeX via pandoc}}
\urlstyle{same} % disable monospaced font for URLs
\usepackage{longtable,booktabs,array}
\usepackage{calc} % for calculating minipage widths
% Correct order of tables after \paragraph or \subparagraph
\usepackage{etoolbox}
\makeatletter
\patchcmd\longtable{\par}{\if@noskipsec\mbox{}\fi\par}{}{}
\makeatother
% Allow footnotes in longtable head/foot
\IfFileExists{footnotehyper.sty}{\usepackage{footnotehyper}}{\usepackage{footnote}}
\makesavenoteenv{longtable}
\usepackage{graphicx}
\makeatletter
\def\maxwidth{\ifdim\Gin@nat@width>\linewidth\linewidth\else\Gin@nat@width\fi}
\def\maxheight{\ifdim\Gin@nat@height>\textheight\textheight\else\Gin@nat@height\fi}
\makeatother
% Scale images if necessary, so that they will not overflow the page
% margins by default, and it is still possible to overwrite the defaults
% using explicit options in \includegraphics[width, height, ...]{}
\setkeys{Gin}{width=\maxwidth,height=\maxheight,keepaspectratio}
% Set default figure placement to htbp
\makeatletter
\def\fps@figure{htbp}
\makeatother
\setlength{\emergencystretch}{3em} % prevent overfull lines
\providecommand{\tightlist}{%
  \setlength{\itemsep}{0pt}\setlength{\parskip}{0pt}}
\setcounter{secnumdepth}{5}
\usepackage{booktabs}
\ifluatex
  \usepackage{selnolig}  % disable illegal ligatures
\fi
\usepackage[]{natbib}
\bibliographystyle{apalike}

\title{R Eğitim}
\author{Melike Dönertaş}
\date{22-03-2021}

\begin{document}
\maketitle

{
\setcounter{tocdepth}{1}
\tableofcontents
}
\hypertarget{bilgilendirme}{%
\chapter*{Bilgilendirme}\label{bilgilendirme}}
\addcontentsline{toc}{chapter}{Bilgilendirme}

22 Mart 2021 tarihinde Bioinforange tarafından düzenlenen Bioinfoconference kapsamında verilen R Eğitim çalıştay materyalini içerir. Çalıştay materyali Melike Dönertaş tarafından hazırlanmıştır, öneri ve düzeltmeler için konu başlığında ``{[}Bioinforange R Eğitim{]}'' kullanarak \href{mailto:donertas.melike@gmail.com}{email} ile ulaşabilirsiniz.

\hypertarget{yardux131m}{%
\section*{Yardım}\label{yardux131m}}
\addcontentsline{toc}{section}{Yardım}

R konusunda karşılaştığınız sorular için öncelikle aldığınız hata ve uyarıları incelemenizi, \texttt{?} operatörünü veya \texttt{help()} komutunu kullanarak fonksiyon, veriseti ve obje yardım sayfalarına bakmanızı, R sıkça sorulan sorular kitapçıklarına sorularınızın dahip olup olmadığını kontrol etmenizi (\href{https://cran.r-project.org/doc/FAQ/R-FAQ.html}{Genel SSS}, \href{https://cran.r-project.org/bin/windows/base/rw-FAQ.html}{Windows'a özgü SSS}, \href{https://cran.r-project.org/bin/macosx/RMacOSX-FAQ.html}{Mac OS'a özgü SSS}), bir paket kullanıyorsanız ve sorunuz bununla ilgiliyse paket yardım ve kullanım örnekleri dosyalarını incelemenizi, ve eğer bu aşamalardan sonra hala cevap bulamadıysanız sorunuzu ``\href{https://stackoverflow.com/help/how-to-ask}{Nasıl güzel soru sorulur?}'' yönergesine göz attıktan sonra \href{https://stackoverflow.com/questions/tagged/r}{Stack Overflow R bölümü}ne sormanızı tavsiye ederim. Alternatif olarak, kodlama konusunda yardımlaşma platformlarına (RSG Türkiye'nin Slack kanalı gibi\footnote{bu şekilde başka platformlar biliyorsanız lütfen bildirin, güncelleyebilirim}) sorularınızı gönderebilirsiniz.

  \bibliography{book.bib}

\end{document}
